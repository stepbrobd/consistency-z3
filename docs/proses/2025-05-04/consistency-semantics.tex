\section{Consistency Semantics}

In Viotti et al.~\cite{viotti2016consistency}, authors provided a
structured overview
of different consistency semantics appeared in distributed systems. Building
upon it's semantic definitions, we propose to use predicate logic derived from
those definitions to model consistency semantics, and reason about the
compositional properties of these semantics with SMT solvers (while Z3
  \cite{demoura2008z3} is used in our implementation, it's easily
  extensible with other
solver backends like CVC5 \cite{barbosa2022cvc5}).

% TODO: summarize operations, axioms, and models

However, our current framework abstracts system-wide behavior from a single
node's perspective, applying constraints to operations primarily within the
scope of a single history and its corresponding abstract executions. While this
focuses the analysis and simplifies reasoning on local consistency properties
from the observing node's perspective, it also means that global behaviors
across multiple nodes need to be modeled separately on different set of symbols
to avoid conflicts in quantified variables. Further, to reason about cross
system behaviors, ``glue'' constraints need to be added to represent data flow
between different systems.

\subsection{Operations}

To capture behaviors in distributed system executions, we model state
transitions on individual nodes using the notion of
\textit{operations}. Similar to
Viotti et al.~\cite{viotti2016consistency}, we define an operation as a tuple
containing: \texttt{proc} (process or node), \texttt{type} (read or
write), \texttt{obj} (the object
being acted upon), \texttt{ival} (input value), \texttt{oval} (output
value), \texttt{stime}
(start time), and \texttt{rtime} (return time). In our SMT encoding,
this tuple is
modeled as a sort, serving as the atomic building block for capturing system
behaviors and state transition constraints.

Partial and total orderings of operations are assigned through relation
functions defined in history and abstract
execution~\cite{burckhardt2014principles}
\cite{viotti2016consistency}. In the implementation, they are defined
as uninterpreted
functions with constraints~\cite{bryant2002modeling}. History
represent the set of all
operations executed in the system, while abstract execution instantiates/refines
the history by specifying which operations are visible to, or are arbitrated
over each other. By embedding these relations in first order logic, we can
leverage SMT solvers to check the satisfiability of a given configuration,
thereby determining whether a particular consistency model holds or if certain
axioms of that model are violated.

When modeling multiple interacting systems or components within a single
analysis context, using distinct symbols for the operations belonging to each
system is crucial. Consistency semantics are typically defined using universally
quantified formulas (e.g., $\forall a, b: \dots$) applied to
relations treated as
uninterpreted functions by the SMT solver (e.g., $\text{``vis''}(a,
b)$, $\text{``so''}(a, b)$).
If the same symbolic operations (like $a, b$) and relation functions
(like $\text{``vis''}$)
were reused across the models of different systems within the same solver query,
the solver would treat them as the \textit{same} entities. This could lead to
unintended interactions and logical conflicts: constraints from one system's
model might implicitly and incorrectly restrict the behavior of operations
logically belonging to another system. By assigning unique symbols to operations
(and potentially relations) associated with different components
(e.g., $a_X$, $b_X$, $\text{``vis''}_X$ for
System X; $a_Y$, $b_Y$, $\text{``vis''}_Y$ for System Y), we maintain
the necessary
logical isolation. This ensures that the axioms and constraints of a specific
consistency model only apply to the operations within that specific system's
scope. This symbolic separation is fundamental for enabling the analysis of
\textit{cross-system} interactions, where specific ``glue''
constraints can then be added
explicitly to relate operations across these distinct symbolic domains, as
demonstrated in our cross-causal consistency modeling.

% FIXME: from readme
%
% these are more relevant to composition, add them here won't make much sense?
%
% Operation: an operation is defined as a tuple of (from
% non-transactional survey):
% proc: the process that issued the operation (used for session
% identification, e.g. same session constraint).
% type: the type of the operation (e.g. read, write. or custom).
% obj: the object the operation is performed on (e.g. a key in a
% key-value store).
% ival: operation input value (e.g. the value to be written, not used
% anywhere yet).
% oval: operation output value (e.g. the value read, not used anywhere yet).
% stime: invocation time of the operation (i.e. the time the
% operation was issued).
% rtime: response time of the operation (i.e. the time the operation
% was completed).
% Think of an operation as a request-response pair, where the request
% is the operation itself and the response is the operation itself
% with the response time set (assuming the delays between the request
% and the response are equal for all operations, the request takes
% (rtime - stime)/2 time to be sent to the handling node and to be
% processed, vice versa for the response).
%
% (Our own interpretation)
% Operation does not contain the node that issued the operation, nor
% the node that handled the operation. It's defined as a generic
% entity that can be issued by any node and handled by any node. To
% add constraints on the nodes that can issue or handle an operation,
% the framework user must define nodes and edges.

\subsection{Axiom}

%
% Relation predicates:
% A set of predicates that define the relations between operations in
% a history (also includes those in AE).
% The predicates are used as base/premise for consistency models
% (implementations in relation.py, history.py,
% abstract_execution.py). In the implementation, the predicates are
% singletons and will be added ad-hoc if any clause refers to them.
%
Similar to how Lamport et al.~\cite{lamport1977proving} defines axioms, they are
operational invariants defined over operations. In our implementation, these
axioms/invariants/relations are declared as uninterpreted functions, where the
SMT solvers are free to interpret them so long as they satisfy the imposed
constraints.

\subsubsection{History}

%
% History (in the actual writing, layout the actual Z3 clauses used
% to encode the relational constraints): a set of operations.
% Add relations between operations within history (returns before,
% same session, session order).
%

A history is a set of operations that have been executed in the system the tool
is modeling~\cite{viotti2016consistency}. Although it is not
specifically defined as
such in our framework, relations are quantified over operations to capture the
\textit{deterministic} orderings of themselves and other operations
in the history. The
operations being quantified over, forms the history of, which is conceptually
identical to the set of operations that have been executed in the system.

Axioms defined for operations within a history:

\begin{itemize}
  \item returns-before: $\text{``rb''} \triangleq \{(a, b) : a, b \in
    H \text{ and } a.\text{``rtime''} < b.\text{``stime''}\}$

  \item same-session: $\text{``ss''} \triangleq \{(a, b) : a, b \in H
    \text{ and } a.\text{``proc''} = b.\text{``proc''}\}$

  \item session-order: $\text{``so''} \triangleq \text{``rb''} \cap
    \text{``ss''}$

  \item same-object: $\text{``ob''} \triangleq \{(a, b) : a, b \in H
    \text{ and } a.\text{``obj''} = b.\text{``obj''}\}$
\end{itemize}

In our implementation, each history relation function (e.g.,
\texttt{History.Relation.returns\_before})
accepts an optional \texttt{symbols} argument. This parameter allows
framework users to
explicitly specify the symbolic operations (e.g., \texttt{["a\_X",
"b\_X"]}) over which
the relation's constraints should be quantified for that specific instance. This
mechanism enables the application of history-based constraints distinctly to
operations belonging to different modeled systems or components, contributing to
the logical isolation necessary for cross-system analysis. If the
\texttt{symbols} argument
is omitted, the functions default to a standard set of symbols (like
\texttt{["a", "b"]}),
applying the constraints to a general set of operations suitable for
single-system analysis.

\subsubsection{Abstract Execution}

%
% Abstract execution (in the actual writing, layout the actual Z3
% clauses used to
% encode the relational constraints): one or more "instantiation" of history. An
% abstract execution is a set of operations (i.e. history) that are further
% constrained by the nondeterminism of the asynchronous environment
% (e.g., message
% delivery order), and implementation-specific constraints (e.g., conflict
% resolution policies) (from non-transactional survey).
%

Abstract executions are instantiated from a given history by specifying which
operations are visible to each other and how they are ordered. Multiple multiple
abstract executions are possible for a single history as observed event oderings
can differ between nodes. AE encode the \textit{non-deterministic} effects of
asynchronous execution environments and implementation-specific constraints
\cite{burckhardt2014principles} \cite{viotti2016consistency}.

Similar to History relations, the Abstract Execution relation functions in our
implementation (e.g., \texttt{AbstractExecution.Relation.visibility})
also accept an
optional \texttt{symbols} argument. This allows framework users to
instantiate AE
relations with constraints quantified over specific symbolic operations relevant
to a particular system or component being modeled. Omitting the argument applies
the relation to the default set of symbols.

Axioms defined for operations within abstract executions:

\begin{itemize}
  \item visibility (vis): visibility definition used in all
    literatures we've reviewed
    are ambiguous in the sense that they don't specify the exact behavior under
    concurrent settings \cite{viotti2016consistency}
    \cite{viotti2016towards} \cite{zhang2018building}
    \cite{ferreira2023antipode}. In our implementation, we
    restructured visibility as a
    binary relation and performed explicit case analysis on all possible
    combinations of read and write operations that can be "visible"
    to each other.
    To achieve visibility, two operations must first fall in one of
    the categories
    in "can-view":
\end{itemize}

% FIXME: way too ugly
% change this into a graph
can-view (cv): $\text{``cv''} \triangleq \{(a, b) : a \in
  H_\text{``rd''}, b \in H_\text{``wr''} \text{ and }
  a.\text{``obj''} = b.\text{``obj''} \text{ and } ($
    $a.\text{``stime''} > b.\text{``rtime''} \text{ // non-concurrent}$
    $\text{or}$
    $(a.\text{``stime''} \leq b.\text{``stime''} \text{ and }
      a.\text{``stime''} \leq b.\text{``rtime''} \text{ and }
      a.\text{``rtime''} \geq b.\text{``stime''} \text{ and }
    a.\text{``rtime''} \geq b.\text{``rtime''})$
    $\text{or}$
    $(a.\text{``stime''} \geq b.\text{``stime''} \text{ and }
      a.\text{``stime''} \leq b.\text{``rtime''} \text{ and }
      a.\text{``rtime''} \geq b.\text{``stime''} \text{ and }
    a.\text{``rtime''} \geq b.\text{``rtime''})$
    $\text{or}$
    $(a.\text{``stime''} \leq b.\text{``stime''} \text{ and }
      a.\text{``stime''} \leq b.\text{``rtime''} \text{ and }
      a.\text{``rtime''} \geq b.\text{``stime''} \text{ and }
    a.\text{``rtime''} \leq b.\text{``rtime''})$
    $\text{or}$
    $(a.\text{``stime''} \geq b.\text{``stime''} \text{ and }
      a.\text{``stime''} \leq b.\text{``rtime''} \text{ and }
      a.\text{``rtime''} \geq b.\text{``stime''} \text{ and }
    a.\text{``rtime''} \leq b.\text{``rtime''})$
$)\}$

In the above encoding, "can-view" is defined as a non-deterministic pairwise
partial ordering that solely depends on time stamps (conceptually, it captures
all cases where reads happened before or after or during writes). And is a set
of operations where, the first element of the tuple is a read and the second
element is a write. The read \textit{can view} (nondeterminism
included) the write if:

\begin{itemize}
  \item (a1) The read-write pair contains non-concurrent operations.

  \item (a2) The read started before the write starts and ended after
    the write ends.

  \item (a3) The read started after the write starts and ended after
    the write ends.

  \item (a4) The read started before the write starts and ended
    before the write ends.

  \item (a5) The read started after the write starts and ended before
    the write ends.
\end{itemize}

\begin{center}
\begin{verbatim}
    a1         a2
  |---|      |---|

             b (rd)
         |------------|

       a3              a4
  |---------|     |----------|

              a5
   |----------------------|
\end{verbatim}
\end{center}

While "can-view" captures the possible visibility between read and write, the
result dependency between them is captured by the "viewed" relation:

$\text{``viewed''} \triangleq \{(a, b) : a \in H_\text{``rd''}, b \in
  H_\text{``wr''} \text{ and } ($
    $a.\text{``oval''} = b.\text{``ival''}$
    $\text{or}$
    $(\exists x \in H_\text{``wr''} \text{ and } (x, b) \in
    \text{``cv''} \text{ and } a.\text{``oval''} = x.\text{``ival''})$
$)\}$

In the encoding above, "viewed" is a non-deterministic pairwise partial ordering
between a write and a read that build atop "can-view". Aside from the timestamps
fall into one of the "can-view" cases, we assigned additional value related
constraints to operations: either the input of the write must match the output
of the read, or, in case of a write happened after or concurrent to the
aforementioned write, viewed relation enforces the output of the read to be
either of writes (but only one can be chosen). In visibility definition, the
transitivity of viewed relation is implicitly enforced.

% actual vis
In our visible/visibility definition, it is defined as a
\textit{deterministic} binary
relationship between any two operations in the history. It also enforces the
transitivity (propagation) and acyclicity (cannot go back in time) of the viewed
relation. We do not enforce transitivity and acyclicity in the viewed relation
as the orderings of concurrent operations are non-deterministic, and viewed
should include all pairs of operations that can be viewed by each other only
from logical time and value equality perspective. The following cases are
considered:

% simplest form: can-view -> viewed -> vis + ar
\begin{itemize}
  \item write-read: $\text{``viewed''} \cap \text{``ar''}$, simply
    viewed (value equivalence) +
    arbitration (preserve the previously viewed ordering as if it's a linearized
    ordering in the first place)

  \item write-write: no constraints
\end{itemize}

% later read tracks earlier read's closest visible write
% FIXME: probably wrong
\begin{itemize}
  \item read-read: $\{(a, b) : a, b \in H_\text{``rd''} \text{ and }
      (\exists x \in H_\text{``wr''} \text{ and } (x, a) \in
    \text{``vis''}) \rightarrow (x, b)\}$,
    previous read needs to track it's closest visible write, then propagate the
    closest visible write to the latest read

  \item read-write: no constraints
\end{itemize}

% FIXME: rework needed
Note that the read-read case is recursively defined, where $a$ and
$b$ are in $\text{``vis''}_\text{``read-read''}$ iff.
there's a write $x$ that is visible to $a$ and $x$ is in
$\text{``vis''}_\text{``read-read''}$ with $b$.

\begin{itemize}
  \item arbitration (ar): an application specific, transitive and
    acyclic relation that
    provides a total order on conflicting operations, ensuring that observed
    executions follow a single coherent timeline.
\end{itemize}

\subsection{Session Guarantees}

In the implementation, the models follow the exact axiomatic definition provide
in Terry et al.~\cite{terry1994session} and Viotti et
al.~\cite{viotti2016consistency}.
Below definitions are copied from the paper.

\subsubsection{Monotonic Reads}

Monotonic reads states that successive reads must reflect a non-decreasing set
of writes. Namely, if a process has read a certain value v from an object, any
successive read operation will not return any value written before v.
Intuitively, a read operation can be served only by those replicas that have
executed all write operations whose effects have already been observed by the
requesting process. In effect, we can represent this by saying that, given three
operations $a, b, c \in H$, if $a \xrightarrow{\text{vis}} b$ and
$b \xrightarrow{\text{so}} c$, where $b$ and $c$ are read operations, then
$a \xrightarrow{\text{vis}} c$, i.e., the transitive closure of vis
and so is included
in vis.

$$
\text{MonotonicReads} \triangleq \forall a \in H, \forall b, c \in
H_\text{rd}: a \xrightarrow{\text{vis}} b \text{ and } b
\xrightarrow{\text{so}} c \Rightarrow a \xrightarrow{\text{vis}} c
$$

\subsubsection{Monotonic Writes}

In a system that ensures monotonic writes a write is only performed on a replica
if the replica has already performed all previous writes of the same session. In
other words, replicas shall apply all writes belonging to the same session
according to the order in which they were issued.

$$
\text{MonotonicWrites} \triangleq \forall a, b \in H_\text{wr}: a
\xrightarrow{\text{so}} b \Rightarrow a \xrightarrow{\text{ar}} b
$$

\subsubsection{Read Your Writes}

Read-your-writes guarantee (also called read-my-writes) requires that a read
operation invoked by a process can only be carried out by replicas that have
already applied all writes previously invoked by the same process.

$$
\text{ReadYourWrites} \triangleq \forall a \in H_\text{wr}, \forall b
\in H_\text{rd}: a \xrightarrow{\text{so}} b \Rightarrow a
\xrightarrow{\text{vis}} b
$$

\subsubsection{Write Follows Read}

Writes-follow-reads, sometimes called session causality, is somewhat the
converse concept of read-your-write guarantee as it ensures that writes made
during the session are ordered after any writes made by any process on any
object whose effects were seen by previous reads in the same session.

$$
\text{WriteFollowsRead} \triangleq \forall a, c \in H_\text{wr},
\forall b \in H_\text{rd}: a \xrightarrow{\text{vis}} b \text{ and }
b \xrightarrow{\text{so}} c \Rightarrow a \xrightarrow{\text{ar}} c
$$

\subsection{PRAM Consistency}

Pipeline RAM (PRAM or FIFO) consistency prescribes that all processes see write
operations issued by a given process in the same order as they were invoked by
that process. On the other hand, processes may observe writes issued by
different processes in different orders. Thus, no global total ordering is
required. However, the writes from any given process (session) must be
serialized in order, as if they were in a pipeline - hence the name.

$$
\text{PRAM} \triangleq \forall a, b \in H: a \xrightarrow{\text{so}}
b \Rightarrow a \xrightarrow{\text{vis}} b
$$

\subsection{Causal Consistency}

Our causal consistency is a combinition of Voitti et
al.~\cite{viotti2016consistency}
and causal memory~\cite{baldoni2002an} extends beyond PRAM and
session guarantees. The
\textit{writes-into} relation~\cite{baldoni2002an} links write
operations directly to the
reads that return their values (at the same memory region). This ensures that if
a read observes a particular write, all subsequent writes in the same session
respect that causal ordering. Formally, it aligns session order (\texttt{so}),
arbitration (\texttt{ar}), visibility (\texttt{vis}), and writes-into
(\texttt{wi}) to maintain
coherent causal histories.

To for operations to be in the \texttt{wi} set: a write \texttt{a}
writes into a read \texttt{b} iff \texttt{b} returns
the value originally written by \texttt{a}, and \texttt{a}/\texttt{b}
to reference the same object
(same memory region). There must be at most one \texttt{a} for each
\texttt{b}, and \texttt{wi} is
acyclic.

If one operation follows another in session order, their relationship in the
abstract execution is further constrained. Specifically, if $(a, b)
\in \text{so}$,
then \texttt{a} must write-into \texttt{b} if \texttt{b} is a read,
and \texttt{a} must be visible and
arbitrated before \texttt{b}. Thus, session order induces a causal
ordering that is
reflected in the relations \texttt{wi}, \texttt{vis}, and \texttt{ar}.

% TODO: vis(a, b) -> (a, b) in "vis"
To incorporate session causility, we conjunct write-follow-reads with
writes-into set: if a read \texttt{b} observes a write \texttt{a}
(i.e. \texttt{vis(a, b)}) and \texttt{b} is
followed by a write \texttt{c} in the same session (\texttt{so(b,
c)}), then \texttt{ar(a, c)} ensures
that the causal dependency introduced by reading \texttt{a}'s value
is respected by the
subsequent write \texttt{c}.

% TODO: need double check
$$
\text{Causal} \triangleq (\forall a, b \in H: (a, b) \in \text{so}
\Leftrightarrow (a, b) \in \text{wi} \cap \text{vis} \cap \text{ar})
\text{ and } \text{WriteFollowsRead}
$$

These conditions together ensure a causal memory model where session order,
observed values, and write sequences are all aligned, causally dependent writes
appear in the correct order from any observer's perspective.

\subsection{Liniearizability}

Although linearizability is widely considered the gold standard for strong
consistency, our initial attempts resulted in a incomplete model. In our draft
encoding, we introduced a single global order to unify visibility and
arbitration for all write operations, and tried to enforce real-time ordering to
ensure that the observed histories comply with returns-before relations
(linearization as operations comes in instead of lazily ordering events when
reads occur~\cite{zhang2018building}). However, the resulting formulas were too
restrictive and did not yield a complete model, as linearizability mandates a
very strong global ordering property that is non-trivial to capture in our
current axiomatic formulation.
