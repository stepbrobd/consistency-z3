\section{Semantic Comparison and Composition}

A core contribution of our framework is the ability to reason not only about
individual consistency semantics in isolation but also about their pairwise
compatibility and compositional properties. This enables exploration of how
different models relate to each other and whether they can be combined to
produce stronger or more application-specific consistency guarantees.

We define compatibility between two consistency semantics $M_1$ and $M_2$ using
an implication-based criterion. $M_1$ is considered compatible with
$M_2$ if the formula $M_1 \Rightarrow M_2$ is valid, i.e., there is no
execution that satisfies all constraints of $M_1$ without also satisfying $M_2$.
We implement this by asserting the negation $\neg (M_1 \Rightarrow M_2)$ and
checking if it is unsatisfiable using our SMT-based approach. If no
counterexample can be found, it implies that $M_1$
refines/subsumes $M_2$. This compatibility check is not symmetric:
$M_1 \Rightarrow M_2$
holding does not necessarily mean that $M_2 \Rightarrow M_1$ also holds.

\begin{table}[htbp]
  \centering
  \caption{Pairwise Compatibility Check ($LHS \Rightarrow RHS$)}
  \label{tab:compatibility}
  \begin{tabular}{|l|c|c|c|c|c|c|}
    \hline
    \diagbox[width=3.0cm]{\textit{LHS}}{\textit{RHS}} &
    \rotatedHeader{PRAM} &
    \rotatedHeader{Monotonic Reads} &
    \rotatedHeader{Monotonic Writes} &
    \rotatedHeader{Read Your Writes} &
    \rotatedHeader{Writes Follow Reads} &
    \rotatedHeader{Linearizability} \\
    \hline
    \textit{PRAM}                & \tableNA   & \tableT    & \tableT
    & \tableT    & \tableF    & \tableF \\
    \textit{Monotonic Reads}     & \tableF    & \tableNA   & \tableF
    & \tableF    & \tableF    & \tableF \\
    \textit{Monotonic Writes}    & \tableF    & \tableF    & \tableNA
    & \tableF    & \tableF    & \tableF \\
    \textit{Read Your Writes}    & \tableF    & \tableF    & \tableF
    & \tableNA   & \tableF    & \tableF \\
    \textit{Writes Follow Reads} & \tableF    & \tableF    & \tableF
    & \tableF    & \tableNA   & \tableF \\
    \textit{Linearizability}     & \tableT    & \tableT    & \tableT
    & \tableT    & \tableT    & \tableNA \\
    \hline
  \end{tabular}
\end{table}

To combine multiple consistency models into a single stronger model, we used
logical conjunctions on each model's constraints. By taking the constraints
representing each model's semantics and forming their conjunction, we derive a
composed model that enforces all included constraints simultaneously. Unlike
compatibility checks, where direction and implication matter, composition is
commutative—adding more models simply layers their constraints on top of one
another. This approach allows incremental composition, users can start from a
base model and iteratively strengthen it by adding new sets of constraints
(either from our implemented models or user-defined constraints) representing
additional consistency guarantees. For example, the conjunction of monotonic
reads, monotonic writes, and read-your-writes can yield PRAM consistency
\cite{brzezinski2004session}. Similarly, layering these and writes-follow-reads
recovers a form of causal consistency \cite{perrin2016causal}.
