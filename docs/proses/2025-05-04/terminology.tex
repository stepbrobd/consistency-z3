\section{Terminology}

% when writing the actual paper, explanations below should be moved
% to pseudocode

Operations: Modeled as an extensible tuples (proc, type, obj,
  ival, oval, stime,
rtime)~\cite{viotti2016consistency} representing atomic actions
in the system.
Additional fields can be added by invoking the sort constructor
with additional
fields.

Axioms/Relations: Predicate logic constraints defined over operations.
\begin{itemize}
  \item History Relations: Deterministic orderings based on time and session
    (returns-before, same-session, session-order, etc.).
  \item Abstract Execution Relations: Non-deterministic orderings
    capturing more complex
    temporal constraints and conflict resolution (visibility, arbitration,
    happens-before, etc.).
\end{itemize}

Consistency Semantics/Models: Specific consistency guarantees
defined as logical
formulas over axioms/relations (e.g., Monotonic Reads, PRAM,
  Causal Consistency,
etc.). Each model enforces safety properties i.e. specific ordering and
visibility constraints on operations within history and abstract executions.

Functions on Models:
\begin{itemize}
  \item Compatibility Check (\texttt{compatible(lhs, rhs)}):
    Determines if one set of
    constraints (lhs) implies another set of constraints (rhs) under a given
    background context (others and global
    Relation.Constraints()). This function
    uses the construct helper function to create a Z3 solver
    instance. construct
    asserts the negation of the implication: Not(Implies(lhs, rhs)). This is
    logically equivalent to lhs And Not(rhs). It also adds the background
    constraints (others and Relation.Constraints()). The
    compatible function then
    calls s.check() on this solver. If s.check() returns z3.unsat
    (unsatisfiable),
    it means there is no model (no counterexample) where lhs is
    true and rhs is
    false. Therefore, whenever lhs holds, rhs must also hold. The
    function returns
    True (compatible). If s.check() returns z3.sat (satisfiable), it means a
    counterexample exists, the implication does not hold
    universally, and the
    function returns False (not compatible).

  \item Composability Check (\texttt{composable(graph, src,
    ...)}): Verifies if a distributed
    system, modeled as a graph (\texttt{graph: nx.MultiDiGraph}),
    can have its components
    interact in a way that respects their individual consistency
    requirements and
    guarantees. The function checks if there exists at least one
    valid way to assign
    specific semantics (from the choices provided in
    needs/provs/cons) such that all
    interactions are consistent. The graph in the function
    parameter composed by
    Nodes (\texttt{Node}) with needs (list of required semantic
    tuples) and provs (list of
    provided semantic tuples). Edges (\texttt{Edge}) connect
    nodes and can have cons (list
    of semantic tuples constraining the interaction). Lists
    represent alternative
    choices (logical or), while tuples within the list represent
    constraints that
    must hold together (logical and). In the implementation, we perform a
    Depth-First Search (DFS) starting from a given source node,
    keeping track of
    visited edges and the accumulated constraints
    (\texttt{path\_premise}) along the current
    path. For each unvisited edge (u,v) from the current node u:
    It iterates through
    all combinations (product) of possible needs from u, provs
    from v, and cons from
    the edge (u, v). For each combination, it performs a
    compatible check: \texttt{compatible(check\_provs,
    compose(check\_needs, check\_ec), path\_premise)}.
    This checks if the chosen provides guarantee of the
    destination (check\_provs) is
    implied by the conjunction (compose) of the chosen needs of the source
    (check\_needs) and the chosen edge constraints (check\_ec),
    considering the
    constraints already accumulated (path\_premise). If a
    compatible combination is
    found that satisfies all unvisited edges between u and v at
    the current step,
    it: Adds the chosen constraints (needs, provs, cons) to the
    result graph.
    Updates the path\_premise by composing the newly satisfied
    constraints. Marks the
    edges as visited. Continues the DFS from the destination node
    v. If the DFS
    eventually visits all edges in the graph, the function
    returns True and the
    result graph showing one valid assignment. If a dead end is
    reached or no
    compatible combination works for an edge, it backtracks,
    removing the choices
    made at that step and trying other combinations. If the DFS
    completes without
    visiting all edges (meaning no fully consistent assignment
    was found), it
    returns False.

  \item Extraction (\texttt{extract(inode, onode, result)}):
    After a successful composable
    check, this function takes the resulting graph from
    composability check (which
    represents ONE valid assignment of constraints) and aggregate
    a single Z3
    constraint (z3.AstRef) that represents the net or equivalent
    semantic guarantee
    of the subgraph reachable from a specific node (inode),
    assuming inode and onode
    are the same. The function performs a DFS (dfs helper) starting from the
    specified inode on this result graph. For each edge (src,
    dst) traversed in the
    DFS, extraction function uses the specific provs constraint
    assigned to src in
    the result graph, uses the specific needs constraint assigned
    to dst in the
    result graph, and retrieves the specific cons constraint
    assigned to the edge
    (src, dst, k) in the result graph. The extract function will
    create a composite
    constraint for this single edge interaction:
    compose(src\_provs, edge\_cons,
    dst\_needs). This represents the conjunction of constraints
    relevant to this
    specific step in the flow. It collects these composite
    constraints from all
    edges reachable from inode in the DFS. Finally, it returns
    the logical and
    (compose) of all collected composite constraints. Simply put,
    axioms/relations
    define constraints on Operations, similarly, extract defines a summary
    constraint on a successfully composed subgraph. If the
    composable check found a
    valid way for components A, B, and C to interact starting
    from A, extract(A, A,
    result) generates a single formula representing the overall
    guarantee provided
    by the A->B->C... chain, as seen from A's perspective, based
    on the specific
    choices made during the composability check.
\end{itemize}
